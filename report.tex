\documentclass[11pt,a4paper,english]{paper} 
\usepackage{fontspec}
\usepackage[utf8]{inputenc}
\usepackage[top=2cm,left=1.25cm,right=1.25cm,bottom=2cm]{geometry}
\usepackage{multirow}
\usepackage{graphicx}
\usepackage{bm}
\usepackage[usenames,dvipsnames]{color}
\usepackage{booktabs}
\usepackage{fancyhdr}
\usepackage[most]{tcolorbox}
\usepackage[authoryear]{natbib}
\usepackage{amsmath}
\usepackage{amssymb}
\usepackage{eucal}
\usepackage[]{minted}
\usepackage{enumitem}
\usepackage{latexsym}
\usepackage{indentfirst}
\usepackage[english]{babel}
\usepackage[autostyle, english = american]{csquotes}
\usepackage{titlesec}

\titleformat*{\section}{\LARGE\bfseries}
\titleformat*{\subsection}{\Large\bfseries}
\titleformat*{\subsubsection}{\large}
\MakeOuterQuote{"}

\setmonofont[Scale=0.75]{Hack NF}

\def \courseNumber {CS3500}
\def \courseName {Operating Systems}
\def \assignmentName {Memory Management}
\def \myName {Akilesh Kannan}
\def \rollNumber {EE18B122}

\pagestyle{fancy}
\fancyhf{}
\rhead{\assignmentName}
\lhead{\courseNumber: \courseName}
\cfoot{\thepage}

\linespread{1.2}

\renewcommand{\familydefault}{\sfdefault} %command to change font to sans-serif

\definecolor{blue(ryb)}{rgb}{0.01, 0.28, 1.0}
\definecolor{green(ryb)}{rgb}{0.28, 1.0, 0.01}
\definecolor{red(ryb)}{rgb}{1.0, 0.01, 0.28}
\definecolor{black(ryb)}{rgb}{0, 0, 0}
\definecolor{gray(ryb)}{rgb}{0.75, 0.75, 0.75}

\newenvironment{colorboxed}[4][gray]{
\begin{tcolorbox}[colback=#1!3!white,colframe=#1(ryb)!50!black,title=\textbf{#2: #3},#4]
}{
\end{tcolorbox}
}

\begin{document} 
\thispagestyle{empty}
\vspace{-4.5cm}

\hspace*{-\parindent}
\begin{minipage}{0.65\textwidth}
{\fontsize{22pt}{10pt}\selectfont\textbf{\assignmentName}}\\[1mm]
\Large
\textit{\courseNumber: \courseName}\\[5mm]
\Large \myName \\
\normalsize \rollNumber \\
\end{minipage}\hfill% push everything to the right
\raisebox{-13mm}{\includegraphics[scale=.28]{logo.pdf}}

\hrule \hrule
\medskip
\vspace{-0.2cm}

\section{Problem 1}
\begin{colorboxed}{Code}{\texttt{kernel/vm.c::vmprint()}}{unbreakable}
    \inputminted[baselinestretch=0.85,firstline=395,breaklines]{c}{kernel/vm.c}
\end{colorboxed}

\subsection{Approach}
\begin{itemize}[noitemsep, nolistsep]
    \item Use a static variable \texttt{level} to store the current page table's level (1, 2 or 3).
    \item Given a page table's base address, iterate through the page table and print all valid entries (check if LSB is set), and their corresponding physical address. Indent with \texttt{..}, printed \texttt{level} times.
    \item Check if there's a next level by checking the permission bits (R, W, X) in the page table entry - if they are all 0s, then there is a next level.
    \item Call \texttt{vmprint()} with argument as page table with base address of current page table entry. Increment \texttt{level} before calling, and decrement \texttt{level} after returning.
\end{itemize}

\newpage

\noindent ak;sdj
%%Beginning References. Don't add any text beyond this.
%%------------------------------------------

%\newpage %sending References to the last page

%\bibliography{paper}
%\bibliographystyle{apalike}
\end{document}
